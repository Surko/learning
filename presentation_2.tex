\documentclass[serif,mathserif]{beamer}
\usepackage{amsmath, amsfonts, epsfig, xspace}
\usepackage{algorithm,algorithmic}
\usepackage{pstricks,pst-node}
\usepackage{multimedia}
\usepackage[normal,tight,center]{subfigure}
\setlength{\subfigcapskip}{-.5em}
\usepackage{beamerthemesplit}
\usepackage{tabularx}

\usepackage[english,czech]{babel}
\usepackage[T1]{fontenc}

%% Použité kódování znaků: obvykle latin2, cp1250 nebo utf8:
\usepackage[utf8]{inputenc}

\usetheme{lankton-keynote}

\author[Lukáš Šurín]{Lukáš Šurín}

\title[Prehľad platformy\hspace{2em}\insertframenumber/\inserttotalframenumber]{iOS - Prehľad platformy }

\date{June 2017}

\institute{Profinit}

\newcolumntype{E}[1]{>{\centering\hsize=#1\hsize\arraybackslash}
X}%

\begin{document}

\maketitle

\begin{frame}
  \frametitle{Outline}
  Prehľad toho čo nás v prezentácii čaká\pause
  \begin{itemize}
  \item Oboznámenie s iOS platformou\pause
   \begin{itemize}
      \item História\pause
      \item Architektúra\pause
      \item Rozpad aplikácie\pause
      \item Swift vs ObjectiveC\pause
  \end{itemize}
  \item Príprava na vývoj pre iOS\pause
  \begin{itemize}
      \item XCode\pause
      \item Emulátor
  \end{itemize}
  \end{itemize}
\end{frame}

\begin{frame}
  \frametitle{Prehľad platformy - História}
  \begin{itemize}
	\item 1985 - založenie firmy NeXT \pause
	\item 1988 - licencovanie Objective-C - stáva sa proprietárnym jazykom\pause
	\item 1989 - NeXTSTEP OS\pause
	\item 1996 - Apple skupuje NeXT\pause
	\item 2001 - Mac OS X (založený na NeXTSTEP)\pause
	\item 2007 - vzniká iPhone OS (založený na Mac OS X)\pause
	\item 2008 - iPhone OS SDK, AppStore\pause
	\item 2010 - iPad\pause
	\item 2010 - iPhone OS premenovaný na iOS\pause
	\item 2014 - WWDC = uvítanie jazyka SWIFT
  \end{itemize}
\end{frame}

\begin{frame}
  \frametitle{Prehľad platformy - História}
  \begin{figure}[h]
	\includegraphics[width=\textwidth]{resources/1/history1.png}
  \end{figure}
  \begin{figure}[h]
	\includegraphics[width=\textwidth]{resources/1/history2.png}
  \end{figure}
  	
\end{frame}

\begin{frame}
  \frametitle{Prehľad platformy - Under Hood - Architektúra}
  \begin{itemize}
	\item platforma postavená nad systémom Darwin - open source OS - hybridná architektúra\pause
	\item takže UNIX
  \end{itemize}
\end{frame}


\begin{frame}
  \frametitle{Prehľad platformy - Under Hood - Architektúra}
  \begin{figure}[h]
	\includegraphics[width=0.75\textwidth]{resources/1/unix.png}
  \end{figure}
\end{frame}

\begin{frame}
  \frametitle{Prehľad platformy - Under Hood - Architektúra}
  \begin{itemize}
	\item platforma postavená nad systémom Darwin - open source OS - hybridná architektúra
	\item takže UNIX
	\item podobné s architektúrou Mac OS X\pause
	\item iOS slúži ako prepojenie aplikácii s hardware\pause
	\item beží na ARM procesoroch\pause
	\item v minulosti 32-bit, teraz 64-bit\pause
	\item má 4 abstraktné vrstvy
  \end{itemize}
\end{frame}

\begin{frame}
  \frametitle{Prehľad platformy - Architektúra}
  \begin{figure}[h]
	\includegraphics[width=\textwidth]{resources/1/architecture1.png}
  \end{figure}
\end{frame}

\begin{frame}
  \frametitle{Prehľad platformy - Rozpad aplikácie}
  \begin{itemize}
	\item má 4 abstraktné vrstvy\pause
	\begin{itemize}
	\item Core OS (low level fičury)\pause
	\begin{itemize}
	\item Accelerate.framework - image processing, LA, ...\pause
	\item CoreBluetooth.framework - interakcie s bluetooth a Low energy módom\pause 
  \end{itemize}	
  \end{itemize}
  \end{itemize}
\end{frame}

\begin{frame}
  \frametitle{Prehľad platformy - Rozpad aplikácie}
  \begin{itemize}
	\item má 4 abstraktné vrstvy
	\begin{itemize}
	\item Core OS (low level fičury)
	\item Core service (high level fičury)\pause
	\begin{itemize}
	\item iCloud storage\pause
	\item CoreFoundation framework - práca so stringom, vlákna, kolekcie, ...\pause
	\item CoreServices framework - accounts.framework, adressbook.framework, core data.framework\pause 
	\item práca s SQL, XML podpora
  \end{itemize}
  \end{itemize}
  \end{itemize}
\end{frame}

\begin{frame}
  \frametitle{Prehľad platformy - Rozpad aplikácie}
  \begin{itemize}
	\item má 4 abstraktné vrstvy
	\begin{itemize}
	\item Core OS (low level fičury)
	\item Core service (high level fičury)
	\item Media\pause
	\begin{itemize}
	\item Grafika - CoreGraphis, CoreAnimation, OpenGL, ...\pause
	\item Audio - práca a podpora zvukových formátov AAC, ALAC, ...\pause 
	\item Video - práca a podpora vizualnych formátov mov, mp4, ...
  \end{itemize}
  \end{itemize}
  \end{itemize}
\end{frame}

\begin{frame}
  \frametitle{Prehľad platformy - Rozpad aplikácie}
  \begin{itemize}
	\item má 4 abstraktné vrstvy
	\begin{itemize}
	\item Core OS (low level fičury)
	\item Core service (high level fičury)
	\item Media
	\item Cocoa Touch\pause
	\begin{itemize}
	\item základné frameworky pre zostavovanie iOS aplikácií\pause
	\item accelerometer, kamera, lokalizácia, ...\pause
	\item podpora klúčových technológií (multi tasking, dotyky, pushky, služby, ...).\pause 
	\item dôležitý UIKit (ui komponenty), auto layout,  storyboards, ... 
  \end{itemize}
  \end{itemize}
  \end{itemize}
\end{frame}

\begin{frame}
  \frametitle{Prehľad platformy - Rozpad aplikácie}
  \begin{figure}[h]
	\includegraphics[width=\textwidth]{resources/1/decompose1.png}
  \end{figure}
\end{frame}

\begin{frame}
  \frametitle{Prehľad platformy - Rozpad aplikácie}
  \begin{figure}[h]
	\includegraphics[width=0.7\textwidth]{resources/1/decompose2.png}
  \end{figure}
\end{frame}

\begin{frame}
  \frametitle{Prehľad platformy - Rozpad aplikácie}
  \begin{figure}[h]
	\includegraphics[width=\textwidth]{resources/1/decompose3.png}
  \end{figure}
\end{frame}

\begin{frame}
  \frametitle{Prehľad platformy - SWIFT vs ObjectiveC}
  \begin{tabularx}{\textwidth}{E{1} E{1}}
  SWIFT & Objective-C \\
  \hline\hline
  friendly syntax & moc urozprávaná syntax \\
  \hline
  type inference & od XCode 8 \_\_auto\_type \\
  \hline
  nepovinné & vynútené bodko čiarky \\
  \hline
  generika ANO & generika ANO od nejakej verzie \\
  \hline
  extensions ANO & extensions ANO \\
  \hline
  first class funkcie & pointery na funkcie \\
  \hline
  bez header súborov & s header súbormi \\
  \hline
  práca podobná C\# & práca s pointermi \\   
  \hline
  jednoduchá práca so stringom & stringy prefixované zavináčmi \\     
  \hline
  vytvorenie objektu konštruktorom Object() & alloc a init \\  
  \hline   
  \end{tabularx}
\end{frame}

\begin{frame}
  \frametitle{Prehľad platformy - SWIFT vs ObjectiveC}
  \begin{figure}[h]
	\includegraphics[width=\textwidth]{resources/1/codeswiftobjc.png}
  \end{figure}
\end{frame}

\begin{frame}
  \frametitle{Príprava na vývoj pre iOS - XCode}
  \begin{itemize}
	\item jediné podporované IDE \pause
	\item obsahuje všetko na kódenie pre iOS (emulátory, testy, developer tooly = najdôležitejší je Instruments)\pause
	\item buildenie je možné aj mimo IDE, no na archivovanie je nutný tak či tak XCodeBuildTools \pause
	\item verzia XCode úzko súvisi s SDK
	\end{itemize}
\end{frame}

\begin{frame}
  \frametitle{Príprava na vývoj pre iOS - Emulátor}
  \begin{itemize}
	\item základna sada pri inštalácii XCode \pause
	\item možnosť pridať iné\pause
	\item v emulatore aj nepodpisane aplikacie\pause
	\item emulátor je veľmi vernou kópiou skutočného zariadenia (veľmi vierohodne sa dajú simulovať chyby) oproti Androidu \pause
	\item v emulátore sa dajú simulovať aj také funkcie ako otlačok prstu, shake gesture, rotácia displeja, externý displej, soft klávesnica, ...
	\end{itemize}
\end{frame}

\begin{frame}
  \frametitle{Questions ???}
\end{frame}
\end{document}
