\documentclass[serif,mathserif]{beamer}
\usepackage{amsmath, amsfonts, epsfig, xspace}
\usepackage{algorithm,algorithmic}
\usepackage{pstricks,pst-node}
\usepackage{multimedia}
\usepackage[normal,tight,center]{subfigure}
\setlength{\subfigcapskip}{-.5em}
\usepackage{beamerthemesplit}

\usepackage[english,czech]{babel}
\usepackage[T1]{fontenc}

%% Použité kódování znaků: obvykle latin2, cp1250 nebo utf8:
\usepackage[utf8]{inputenc}

\usetheme{lankton-keynote}

\author[Lukáš Šurín]{Lukáš Šurín}

\title[Úvod\hspace{2em}\insertframenumber/\inserttotalframenumber]{iOS - Úvod }

\date{June 2017}

\institute{Profinit}

\begin{document}

\maketitle

\section{Introduction}  % add these to see outline in slides

\begin{frame}
  \frametitle{O čom to bude}
  \begin{itemize}
  \item Oboznámenie s iOS platformou (veľmi krátko základné teoretické poznatky, história, ...)\pause
  \item Rozbehnutie prostredia na vývoj pre iOS \pause
  \item Štruktúra projektu \pause
  \item Život appky \pause
  \item Don't try to implement what is already implemented, ala. knižnice na iOS \pause
  \item Vývoj na iOS\pause
  \begin{itemize}
  \item UI\pause
  \item Networking\pause
  \item ...
  \end{itemize}
  \end{itemize}
\end{frame}

\begin{frame}
  \frametitle{Odkazy}
  \begin{itemize}
  \item iOS Developer Guide - https://developer.apple.com/library/content/navigation/ 
  \item Tutorials point - https://www.tutorialspoint.com/ios/
  \item Online kurzy - Udacity = https://www.udacity.com/
  \item Nápomocné stránky
  \begin{itemize}
  \item plno návodov a vysvetlivok = {\footnotesize https://www.raywenderlich.com}
  \item rôzne srandy = {\footnotesize www.appcoda.com (sprite kit, research kit, ...) }
  \item etc.
  \end{itemize}
  \end{itemize}
\end{frame}

\begin{frame}
  \frametitle{Otázky}
\end{frame}
\end{document}
