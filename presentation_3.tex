\documentclass[serif,mathserif]{beamer}
\usepackage{amsmath, amsfonts, epsfig, xspace}
\usepackage{algorithm,algorithmic}
\usepackage{pstricks,pst-node}
\usepackage{multimedia}
\usepackage[normal,tight,center]{subfigure}
\setlength{\subfigcapskip}{-.5em}
\usepackage{beamerthemesplit}
\usepackage{tabularx}

\usepackage[english,czech]{babel}
\usepackage[T1]{fontenc}

%% Použité kódování znaků: obvykle latin2, cp1250 nebo utf8:
\usepackage[utf8]{inputenc}

\usetheme{lankton-keynote}

\author[Lukáš Šurín]{Lukáš Šurín}

\title[Štruktúra appky\hspace{2em}\insertframenumber/\inserttotalframenumber]{iOS - Štruktúra appky }

\date{June 2017}

\institute{Profinit}

\newcolumntype{E}[1]{>{\centering\hsize=#1\hsize\arraybackslash}
X}%

\begin{document}

\maketitle

\begin{frame}
  \frametitle{Outline}
  Prehľad toho čo nás v prezentácii čaká\pause
  \begin{itemize}
  \item Zoznámenie s IDE\pause
  \item Zakladanie nového projektu\pause
  \item Štruktúra projektu\pause
   \begin{itemize}
      \item Projekt\pause
      \item Target\pause
      \item Nastavenia targetu\pause
      \item Nastavenie buildu\pause
      \item Assety\pause
      \item UI (xib, storyboard)\pause
      \item kód\pause
  \end{itemize}
  \end{itemize}
\end{frame}

\begin{frame}
  \frametitle{Zoznámenie s IDE}
 \begin{figure}[h]
	\includegraphics[width=\textwidth]{resources/2/IDE1.png}
  \end{figure}
\end{frame}

\begin{frame}
  \frametitle{Zoznámenie s IDE}
 \begin{figure}[h]
	\includegraphics[width=\textwidth]{resources/2/IDE2.png}
  \end{figure}
\end{frame}

\begin{frame}
  \frametitle{Zoznámenie s IDE}
 \begin{figure}[h]
	\includegraphics[width=\textwidth]{resources/2/IDE3.png}
  \end{figure}
\end{frame}

\begin{frame}
  \frametitle{Zoznámenie s IDE}
 \begin{figure}[h]
	\includegraphics[width=\textwidth]{resources/2/IDE4.png}
  \end{figure}
\end{frame}

\begin{frame}
  \frametitle{Zoznámenie s IDE}
 \begin{figure}[h]
	\includegraphics[width=\textwidth]{resources/2/IDE5.png}
  \end{figure}
\end{frame}

\begin{frame}
  \frametitle{Zoznámenie s IDE}
 \begin{figure}[h]
	\includegraphics[width=0.60\textwidth]{resources/2/IDE6.png}
  \end{figure}
\end{frame}

\begin{frame}
  \frametitle{Zoznámenie s IDE}
 \begin{figure}[h]
	\includegraphics[width=\textwidth]{resources/2/IDE7.png}
  \end{figure}
\end{frame}

\begin{frame}
  \frametitle{Zoznámenie s IDE}
 \begin{figure}[h]
	\includegraphics[width=0.7\textwidth]{resources/2/IDE8.png}
  \end{figure}
\end{frame}

\begin{frame}
  \frametitle{Zoznámenie s IDE}
 \begin{figure}[h]
	\includegraphics[width=\textwidth]{resources/2/IDE9.png}
  \end{figure}
\end{frame}

\begin{frame}
  \frametitle{Zoznámenie s IDE}
 \begin{figure}[h]
	\includegraphics[width=\textwidth]{resources/2/IDE10.png}
  \end{figure}
\end{frame}

\begin{frame}
  \frametitle{Zoznámenie s IDE}
 \begin{figure}[h]
	\includegraphics[width=\textwidth]{resources/2/IDE11.png}
  \end{figure}
\end{frame}

\begin{frame}
  \frametitle{Zoznámenie s IDE}
 \begin{figure}[h]
	\includegraphics[width=\textwidth]{resources/2/IDE12.png}
  \end{figure}
\end{frame}

\begin{frame}
  \frametitle{Zoznámenie s IDE}
 \begin{figure}[h]
	\includegraphics[width=\textwidth]{resources/2/IDE13.png}
  \end{figure}
\end{frame}


\begin{frame}
  \frametitle{Zakladanie nového projektu}
 \begin{figure}[h]
	\includegraphics[width=\textwidth]{resources/2/newproject1.png}
  \end{figure}
\end{frame}

\begin{frame}
  \frametitle{Zakladanie nového projektu}
  \begin{itemize}
  \item otvorenie playgroundu = sandboxu, v ktorom sa dajú vyskúšať malé funkčné koncepty\pause
  \item vytvorenie projektu\pause
  \item checkout existujúceho projektu\pause
  \item otvorenie už existujúceho projektu
  \end{itemize}
\end{frame}

\begin{frame}
  \frametitle{Zakladanie nového projektu}
 \begin{figure}[h]
	\includegraphics[width=\textwidth]{resources/2/newproject2.png}
  \end{figure}
\end{frame}

\begin{frame}
  \frametitle{Zakladanie nového projektu}
  \begin{itemize}
  \item krásny prehľad toho čo všetko XCode dokáže\pause
  \item XXXApplication - rôzne typy aplikačných šablón\pause
  \item framework, library\pause
  \item watchOS, macOS, tvOS, CrossPlatform
  \end{itemize}
\end{frame}

\begin{frame}
  \frametitle{Zakladanie nového projektu}
 \begin{figure}[h]
	\includegraphics[width=\textwidth]{resources/2/newproject3.png}
  \end{figure}
\end{frame}

\begin{frame}
  \frametitle{Zakladanie nového projektu}
  \begin{itemize}
  \item meno produktu - ako sa bude zobrazovať na zariadení\pause
  \item voľba teamu - kto a akým spôsobom bude aplikácia vystavená\pause
  \item názov organizácie - informačný text, ktorý sa zobrazuje v rámci projektu\pause
  \item identifikátor - prefix pre bundle ID \pause
  \item bundle identifikátor - unikátny string (povedzme package)\pause
  \item jazyk - Swift, ObjectiveC\pause
  \item voľba zariadení - iPhone, iPad, obe\pause
  \item core data - priama podpora databázového riešenia pre iOS (rovno aj s migračným riešením)\pause
    \item unit testy a ui testy - vytvorenie nových targetov na rôzne druhy testov
  \end{itemize}
\end{frame}

\begin{frame}
  \frametitle{Štruktúra projektu - Projekt}
 \begin{figure}[h]
	\includegraphics[width=\textwidth]{resources/2/project1.png}
  \end{figure}
\end{frame}

\begin{frame}
  \frametitle{Štruktúra projektu - Projekt}
  \begin{itemize}
  \item spoločné nastavenia pre celý projekt\pause
  \item funguje tu dedičnosť nastavení
  \end{itemize}
\end{frame}

\begin{frame}
  \frametitle{Štruktúra projektu - Target}
 \begin{figure}[h]
	\includegraphics[width=\textwidth]{resources/2/target1.png}
  \end{figure}
\end{frame}

\begin{frame}
  \frametitle{Štruktúra projektu - Target}
  \begin{itemize}
  \item jemnejšie nastavenie pre konkrétne potreby\pause
  \item nastavenia overriduju to čo je v projekte\pause
  \item každý target môže byť podpisovaný iným teamom\pause
  \item rieši sa tým problémy súvisiace s dodávaním aplikácií od iných firiem (android build type)\pause
    \item každý vytvorený kód môže alebo nemusí patriť danému targetu\pause
  \item každý target môže mať ľubovoľne veľa schém (jemnejšie delenie)\pause
  \item medzi targety patria aj testovacie scenáre
  \end{itemize}
\end{frame}

\begin{frame}
  \frametitle{Štruktúra projektu - Nastavenia targetu}
  \begin{itemize}
  \item veľa podobného s vytváraním nového projektu\pause
  \item unikátne bundleID\pause
  \item verzia, číslo buildu\pause
  \item zariadenia, orientácie \pause
  \item ikony a splash obrazovky\pause
  \item nalinkované knižnice, frameworky
  \end{itemize}
\end{frame}

\begin{frame}
  \frametitle{Štruktúra projektu - Nastavenia buildu}
  \begin{itemize}
  \item väčšina vecí sa dá nastaviť priamo tu\pause
  \item dajú sa definovať vlastné konzolové premenné\pause
  \item architektúra \pause
  \item vyšperkovanie linkeru, odkiaľ sa berú knižnice, minifikovanie,  \pause
  \item lepšia administrácia podpisovania \pause
  \item nastavenie statického analyzovania kódu \pause
  \item ako sa bude kód optimalizovať \pause
  \item aká verzia jazyka sa použjie
  \end{itemize}
\end{frame}

\begin{frame}
  \frametitle{Štruktúra projektu - Assety}
  \begin{itemize}
  \item ikona appky
  \end{itemize}
\end{frame}

\begin{frame}
  \frametitle{Štruktúra projektu - Assety}
 \begin{figure}[h]
	\includegraphics[width=\textwidth]{resources/2/assets1.png}
  \end{figure}
\end{frame}

\begin{frame}
  \frametitle{Štruktúra projektu - Assety}
  \begin{itemize}
  \item ikona appky
  \item ako aj ostatné obrázky  
  \end{itemize}
\end{frame}

\begin{frame}
  \frametitle{Štruktúra projektu - Assety}
 \begin{figure}[h]
	\includegraphics[width=0.5\textwidth]{resources/2/assets2.png}
  \end{figure}
\end{frame}

\begin{frame}
  \frametitle{Štruktúra projektu - Assety}
 \begin{figure}[h]
	\includegraphics[width=\textwidth]{resources/2/assets3.png}
  \end{figure}
\end{frame}

\begin{frame}
  \frametitle{Štruktúra projektu - Assety}
  \begin{itemize}
  \item ikona appky
  \item ako aj ostatné obrázky
  \item nastavenie pre rôzne zariadenia (každé má iné požiadavky 2x, 3x vs 1x, 2x)\pause
  \item veľkosti treba dodržať\pause
  \item nie vždy sú nutné všetky tri (závisí na požiadavkách)\pause
  \item dodanie by mal zaručiť grafik, dizajnér\pause
  \item k obrázkom sa dajú nastaviť dodatočné možnosti (kvalita renderu, alignment, ...)
    \item obrázky sú pri spustení v bundloch
  \end{itemize}
\end{frame}

\begin{frame}
  \frametitle{Štruktúra projektu - UI (xib, storyboard)}
  \begin{itemize}
  \item správne miesto nie je predom určené \pause
  \item xib - jeden view \pause
  \item storyboard - jeden flow \pause
  \item vo veľa prípadoch je dobré sa vyhnúť storyboardom (niekedy aj xibom) a siahnuť po definovaní UI v kóde
  \end{itemize}
\end{frame}

\begin{frame}
  \frametitle{Štruktúra projektu - UI (xib, storyboard)}
 \begin{figure}[h]
	\includegraphics[width=0.7\textwidth]{resources/2/UI1.png}
  \end{figure}
\end{frame}

\begin{frame}
  \frametitle{Štruktúra projektu - UI (xib, storyboard)}
 \begin{figure}[h]
	\includegraphics[width=0.8\textwidth]{resources/2/UI2.png}
  \end{figure}
\end{frame}

\begin{frame}
  \frametitle{Štruktúra projektu - UI (xib, storyboard)}
 \begin{figure}[h]
	\includegraphics[width=0.6\textwidth]{resources/2/UI3.png}
  \end{figure}
\end{frame}

\begin{frame}
  \frametitle{Štruktúra projektu - UI (xib, storyboard)}
 \begin{figure}[h]
	\includegraphics[width=0.4\textwidth]{resources/2/UI4.png}
  \end{figure}
\end{frame}

\begin{frame}
  \frametitle{Štruktúra projektu - UI (xib, storyboard)}
 \begin{figure}[h]
	\includegraphics[width=0.4\textwidth]{resources/2/UI5.png}
  \end{figure}
\end{frame}

\begin{frame}
  \frametitle{Štruktúra projektu - UI (xib, storyboard)}
 \begin{figure}[h]
	\includegraphics[width=0.4\textwidth]{resources/2/UI6.png}
  \end{figure}
\end{frame}

\begin{frame}
  \frametitle{Štruktúra projektu - UI (xib, storyboard)}
 \begin{figure}[h]
	\includegraphics[width=0.4\textwidth]{resources/2/UI7.png}
  \end{figure}
\end{frame}

\begin{frame}
  \frametitle{Štruktúra projektu - UI (xib, storyboard)}
 \begin{figure}[h]
	\includegraphics[width=0.4\textwidth]{resources/2/UI8.png}
  \end{figure}
\end{frame}

\begin{frame}
  \frametitle{Štruktúra projektu - UI (xib, storyboard)}
 \begin{figure}[h]
	\includegraphics[width=\textwidth]{resources/2/UI9.png}
  \end{figure}
\end{frame}

\begin{frame}
  \frametitle{Štruktúra projektu - kód}
  \begin{itemize}
  \item správne miesto nie je predom určené \pause
  \item dosti závisí od zvolenej architektúry (MVC, MVP, MVVM, VIPER, ...) \pause  
  \item kód pre lepšiu prehľadnosť v imaginárnych zložkách (v adresári projektu zložky nie sú) \pause
  \item kód môže byť ako vo SWIFT tak aj v ObjectiveC \pause  
  \item vytváranie pomocou CMD + N
  \end{itemize}
\end{frame}

\begin{frame}
  \frametitle{Questions ???}
\end{frame}
\end{document}
