\documentclass[serif,mathserif]{beamer}
\usepackage{amsmath, amsfonts, epsfig, xspace}
\usepackage{algorithm,algorithmic}
\usepackage{pstricks,pst-node}
\usepackage{multimedia}
\usepackage[normal,tight,center]{subfigure}
\setlength{\subfigcapskip}{-.5em}
\usepackage{beamerthemesplit}
\usepackage{tabularx}

\usepackage[english,czech]{babel}
\usepackage[T1]{fontenc}

%% Použité kódování znaků: obvykle latin2, cp1250 nebo utf8:
\usepackage[utf8]{inputenc}

\usetheme{lankton-keynote}

\author[Lukáš Šurín]{Lukáš Šurín}

\title[Život appky\hspace{2em}\insertframenumber/\inserttotalframenumber]{iOS - Život appky }

\date{June 2017}

\institute{Profinit}

\newcolumntype{E}[1]{>{\centering\hsize=#1\hsize\arraybackslash}
X}%

\begin{document}

\maketitle


\begin{frame}
  \frametitle{Outline}
  Prehľad toho čo nás v prezentácii čaká\pause
  \begin{itemize}
  \item štruktúra aplikácie a hlavné komponenty\pause
  \item vykonávanie hlavného cyklu\pause
  \item stavy aplikácie + životný cyklus\pause
  \end{itemize}
\end{frame}

\begin{frame}
  \frametitle{Štruktúra aplikácie}
  \begin{itemize}
  \item UIApplicationMain funkcia pri štarte\pause
  \item prichystá klúčové objekty\pause
  \item v srdci je UIApplication - interakcie medzi systémom a appkou \pause
  \item základné časti na obrázku   
  \end{itemize}
\end{frame}

\begin{frame}
  \frametitle{Štruktúra aplikácie}
 \begin{figure}[h]
	\includegraphics[width=\textwidth]{resources/3/appstruct1.png}
  \end{figure}
\end{frame}

\begin{frame}
  \frametitle{Štruktúra aplikácie}
  \begin{itemize}
  \item UIApplication - managovanie hlavného cyklu, signalizuje svoj delegát o rôznych udalostiach\pause
  \item AppDelegate - srdce kódu, ktoré píšeme už my \pause
  \item Model - jednoduché objekty uchovavajúce info o entitách \pause
  \item ViewController - managovanie toho čo prezentujeme na obrazovke (jedna obrazovka a prechody) \pause
    \item UIWindow - ma na starosť koordinovanie a prezentovanie obrazoviek, býva väčšinou jeden \pause
    \item Views a UI - vizuálna reprezentácia obsahu aplikácie, pohlcuje eventy a posiela do controlleru, definované v UIKit
  \end{itemize}
\end{frame}

\begin{frame}
  \frametitle{Vykonávanie hlavného cyklu a udalosti}
  \begin{itemize}
  \item spracováva všetky udalosti \pause
  \item začne pri spustení aplikácie \pause
  \item spustený na hlavnom vlákne \pause
  \item pri interakcii vznikajú udalosti, ktoré idu do appky skrze špeciálny port \pause
    \item udalosti sú zhromaždované vo fronte udalostí \pause
    \item odtiaľ jeden po druhom sú spracované hlavným cyklom \pause
    \item medzi udalosti patria dotyky, zatrasenie, accelerometer, gyroskop, lokácia, ... 
  \end{itemize}
\end{frame}

\begin{frame}
  \frametitle{Vykonávanie hlavného cyklu a udalosti}
 \begin{figure}[h]
	\includegraphics[width=\textwidth]{resources/3/appstruct2.png}
  \end{figure}
\end{frame}

\begin{frame}
  \frametitle{Stavy aplikácie + životný cyklus}
  \begin{itemize}
  \item aplikácia je vždy v každom okamžiku v určitom stave: \pause
  \item aplikácia nebeží - nebola spustená alebo bola terminovaná \pause
  \item aplikácia je neaktívna - je v popredí ale nepríjma žiadne udalosti (v takomto stave iba chvílu) \pause
  \item aplikácia je aktívna - je v popredí a príjma udalosti \pause
  \item aplikácia je v pozadí -  \pause
  \item aplikácia je v suspendovaná - nevykonáva sa žiadny kód, pri low memory môže takéto aplikácie odstrániť
    \item väčšina prechodov sa dá zachytiť v AppDelegate
  \end{itemize}
\end{frame}

\begin{frame}
  \frametitle{Stavy aplikácie + životný cyklus}
 \begin{figure}[h]
	\includegraphics[width=0.7\textwidth]{resources/3/appstruct3.png}
  \end{figure}
\end{frame}


\begin{frame}
  \frametitle{Questions ???}
\end{frame}
\end{document}
